\documentclass[a4paper]{article}
\usepackage[english]{babel}
\usepackage{hyperref}
\usepackage{lmodern}
\usepackage{fancyhdr}
\usepackage{multicol}
 
\pagestyle{fancy}
\fancyhf{}
\lfoot{\small References are available on request}
\rfoot{\small \the\year\hspace{6pt}Artem Chepurnoy}
\renewcommand{\headrulewidth}{0pt}

\newcommand\myphone{+3809339 $[7^2-1]$ 407}
\newcommand\myaddress{23 august, Kharkiv, \MakeUppercase{Ukraine}}

%\input{private}

\begin{document}
	\begin{center}
		{\huge\bfseries Android Developer} \\[1em]
		{\Large Artem Chepurnoy} \\[1em]
		\myaddress
	\end{center}
	\vspace{1em}
	\textit{mail@artemchep.com} \hspace*{\fill} \textit{\myphone} 

	\section*{Personal Statement}
	A versatile and professional software developer with a commitment to and experience of developing innovative and complex software solutions. 
	More than six years of experience in developing Android apps. \\[1em]
	I am Recognized Developer and Themer at \href{https://forum.xda-developers.com/member.php?u=3685328}{XDA-Developers} from 2013 to present time.

	\section*{Skills}
	\begin{multicols}{2}
	\begin{itemize}
		\item \textbf{Programming} with Java, Kotlin, C\#, Python, Delphi etc.
		\item \textbf{Android apps} with Java, Kotlin.
		\item \textbf{Databases} with SQL, NoSQL.
		\item \textbf{Web apps} with JavaScript, HTML/CSS, ReactJS, Angular.
		\item \textbf{English} with intermediate level.
	\end{itemize}
	\end{multicols}

	\section*{Projects}
	\begin{itemize}
		\item \href{http://artemchep.com/horario/}{\textbf{Horario}} \textit{(November 2016 -- Present)} --- Learning platform for students that aims to simplify creating, distributing and grading assignments as well as improving communication between students and teachers. \\[0.2em]
		{\footnotesize \textit{\textbf{Author} \hfill Open-source, Kotlin, Firebase, Java, Gradle}} \\[0.2em]
		Designed and implemented the concept, designed the model of document-oriented database for platform, designed the logo, designed and implemented architecture to support both phones and tablets, maintainer of repository. \\[0.2em]
		Horario was fully rewritten on Kotlin and Firebase Firestore instead of Java and Firebase Realtime Database. 
		
		\item \href{http://artemchep.com/acdisplay/}{\textbf{AcDisplay}} \textit{(December 2013 -- July 2015)} --- Beautiful app for handling incoming notifications. \\[0.2em]
		{\footnotesize \textit{\textbf{Author} \hfill Open-source, Java, Python, Gradle}} \\[0.2em]
		Pioneered custom Ambient display category of apps, designed and implemented the concept, developed touch-forwarding module, designed the logo, maintainer of repository.\\[0.2em]
		AcDisplay has been installed over a three million times on Google Play.

		\item \href{http://getcorgi.com/}{\textbf{Corgi for Feedly}} \textit{(March 2015 – August 2017)} --- Your personal magazine right on the lock screen. \\[0.2em]
		{\footnotesize \textit{\textbf{Software Engineer} \& \textbf{Designer} at Corgi for Feedly Lock Screen \hfill Java}} \\[0.2em]
		Designed and developed the Android Keyguard and Notifications modules, implemented the graphical user interface, helped to design the concept of \textit{extended} lock screen, helped to maintain git-repository of the project.
		
		\item \href{https://play.google.com/store/apps/details?id=com.achep.fifteenpuzzle}{\textbf{FifteenPuzzle}} \textit{(January 2013 -- December 2013)} --- Best fifteen puzzle for Android. \\[0.2em]
		{\footnotesize \textit{\textbf{Author} \hfill Open-source, Java, OpenGL ES2}} \\[0.2em]
		Designed and implemented the concept, implemented rendering via OpenGL, developed methods for handling touches, animated the puzzle board.
		
		\item \href{https://github.com/AChep/SudokuSolver}{\textbf{SudokuSolver}} \textit{(January 2015 -- April 2015)} --- Simple sudoku-solver that can be used to solve or help to generate sudokus. \\[0.2em]
		{\footnotesize \textit{\textbf{Author} \hfill Open-source, Python 3}} \\[0.2em]
		Implemented rules such as <<\href{http://www.sudokuwiki.org/Getting_Started}{Hidden Singles}>>, <<\href{http://www.sudokuwiki.org/Hidden_Candidates}{Hidden Candidates}>>, <<\href{http://www.sudokuwiki.org/Naked_Candidates#NP}{Naked Pairs}>> and <<\href{http://www.sudokuwiki.org/intersection_removal}{Intersection Removal}>>, developed two variants of the script: the one that only uses rules and the one that uses brute-force if stuck.
		
		\item \href{https://forum.xda-developers.com/showthread.php?t=1188486}{\textbf{GingerDX}} \textit{(July 2012 -- May 2013)} --- Fast, smooth and lightweight CM7 based ROM with some special features. \\[0.2em]
		{\footnotesize \textit{\textbf{Software Engineer} \& \textbf{Designer} \hfill Open-source, Java}} \\[0.2em] 
		Added motion gestures to alarm app (merged into CM later on), modified system UI to be looking like Android 4.x, helped to write digitizer module to bring partial multitouch support.  

		\item Multiple other apps and custom ROMs.
	\end{itemize}
	\subsection*{Profiles}
	\begin{tabular}{@{}ll}
		GitHub & \href{https://github.com/AChep}{\texttt{github.com/AChep}} \\
		LinkedIn & \href{https://www.linkedin.com/in/artemchep/}{\texttt{linkedin.com/in/artemchep}} \\
		Stack Overflow & \href{https://stackoverflow.com/users/1408535/achep}{\texttt{stackoverflow.com/users/1408535/achep}} \\
		XDA-Developers & \href{https://forum.xda-developers.com/member.php?u=3685328}{\texttt{forum.xda-developers.com/member.php?u=3685328}} \\
	\end{tabular}

	\section*{Education}
	\textbf{2014 National Technical University <<Kharkiv polytechnic institute>> Software Engineering and Management Information Technologies}. \\
	Modules included 
	\begin{multicols}{2}
	\begin{itemize}
		\item \textbf{English} \textit{(A)}
		\item \textbf{Calculus} \textit{(A)}
		\item \textbf{Fuzzy Technologies} \textit{(A)}
		\item \textbf{Database Modeling} \textit{(C)}
		\item \textbf{Programming} \textit{(A)}
		\item \textbf{Software Testing} \textit{(A)}
		\item \textbf{Software Modeling and Analysis} \textit{(A)}
		\item \textbf{Requirements analysis} \textit{(A)}
	\end{itemize}
	\end{multicols}
	
	\section*{Certificates}
	\subsection*{Cambridge Certification Authority}
	\begin{itemize}
		\item \textbf{Java Level 1}, completed on 1st March 2018
		\item \textbf{Python Level 1}, completed on 1st March 2018
	\end{itemize}

	\section*{Achievements}
	\begin{itemize}
		\item \textbf{IT-EUREKA! HACKATHON} with Horario platform.
		\item \textbf{Battle of Universities. Startup.Network} with Shop Sales --- Android application that allows users to subscribe on categories of products, view map of nearby sales.
	\end{itemize}
	A prizewinner of multiple regional olympiads in Programming, Information Technology, Astronomy and Math. 

\end{document}
