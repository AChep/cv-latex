% !TEX TS-program = lualatex

\documentclass[a4paper]{article}
\usepackage[english]{babel}
\usepackage{fontawesome}
\usepackage{hyperref}
\usepackage{lmodern}
\usepackage{fancyhdr}
\usepackage{multicol}
\usepackage{xcolor}

\usepackage{datenumber}
\usepackage{calc}

\newcounter{datetoday}
\newcounter{diffyears}
\newcounter{diffmonths}
\newcounter{diffdays}

% based on the 
% https://tex.stackexchange.com/a/14519/204629
\newcommand{\difftoday}[3]{%
      \setmydatenumber{datetoday}{\the\year}{\the\month}{\the\day}%
      \setmydatenumber{diffdays}{#1}{#2}{#3}%
      \addtocounter{diffdays}{-\thedatetoday}%
      \ifnum\value{diffdays}>0
        \def\diffbefore{in }%
      \else
        \def\diffbefore{}%
        \setcounter{diffdays}{-\value{diffdays}}%
      \fi
      \setcounter{diffyears}{\value{diffdays}/365}%
      \setcounter{diffdays}{\value{diffdays}-365*\value{diffyears}}%
      \setcounter{diffmonths}{\value{diffdays}/30}%
      \setcounter{diffdays}{\value{diffdays}-30*\value{diffmonths}}%
      %
      \diffbefore
      \ifnum\value{diffyears}=0
      \else
        \ifnum\value{diffyears}>1
            \thediffyears\space years,
        \else
            \thediffyears\space year,
        \fi
      \fi
}
 
\pagestyle{fancy}
\fancyhf{}
\lfoot{\small References are available on request}
\rfoot{\small \the\year\hspace{6pt}Artem Chepurnoy}
\renewcommand{\headrulewidth}{0pt}

\newcommand\myphone{+3809339 $[7^2-1]$ 407}
\newcommand\myaddress{23 august, Kharkiv, \MakeUppercase{Ukraine}}

%\input{private}

\begin{document}
	\begin{center}
		{\huge\bfseries Android Developer} \\[1em]
		{\Large Artem Chepurnoy} \\[1em]
		\faMapMarker \quad \myaddress
	\end{center}
	\vspace{1em}
	\textit{\faPaperPlane \quad mail@artemchep.com} \hspace*{\fill} \textit{\faPhone \quad \myphone} 

	\section*{Personal Statement}
	A versatile and professional software developer with a commitment to and experience of developing innovative and complex software solutions. 
	More than \difftoday{2011}{08}{19} of experience in developing software.

	\section*{Skills}
	\subsection*{General}
	\begin{multicols}{2}
	\begin{itemize}
		\item \textbf{Programming} with Kotlin, Java, Dart, Python, JavaScript, etc.
		\item \textbf{Native Android apps} with Kotlin, Java, \textbf{multiplatform apps} with Kotlin \textbf{crossplatform apps} with Dart \& Flutter, JavaScript.
		\item \textbf{Building} with Gradle, Maven, Webpack, \textbf{CI/CD} with GitHub actions, TeamCity, Fastlane.
		\item \textbf{Designing apps} with Object-Oriented and Functional programming principles, Hexagonal and Clean architecture, MVP, MVVM, MVI patterns and more.
		\item \textbf{Databases} with SQL, NoSQL.
		\item \textbf{Web Servers} with ktor, NodeJS.
		\item \textbf{Version control} with Git.
		\item \textbf{English} with upper intermediate (B2) level.
	\end{itemize}
	\end{multicols}
	\subsection*{Android}
	\begin{multicols}{2}
	\begin{itemize}
		\item \textbf{Advanced at} Kotlin, Reactive programming, Architecture design.
		\item \textbf{Strong at} Custom Views, Background/Foreground Services, Navigation, JetPack, Multithreading, Memory management, Gradle, Phone \& Tablet support, CI, DI, etc.
		\item \textbf{Good at} Video streaming, Databases, Bluetooth, Wear OS, OpenGL ES, etc.
	\end{itemize}
	\end{multicols}
	
	{ \footnotesize
	A few libraries I often use: androidx.*, coroutines, arrow-kt, kodein-di, okhttp, retrofit, moshi, room, glide, exo-player.       
	}
	
	\section*{Experience}
	\begin{tabular}{@{}l@{\enspace--\enspace}ll}
		\textbf{2020 Aug.} & Present & Android Department Tech Lead at \href{https://chisw.com}{\textbf{CHI Software}}. \\
	\end{tabular} \\[0.2em]
	Interviewing Android \& Flutter developers, Managing the Department \& Developers development plans, Maintaining internal projects, Technical \& Performance reviews, Code reviews, Tech-talks, Estimating projects, Mentoring and App development. \\[0.4em]
	\begin{tabular}{@{}l@{\enspace--\enspace}ll}
		\textbf{2018 Feb.} & \textbf{2020 Aug.} & Android Developer at \href{https://chisw.com}{\textbf{CHI Software}}. \\
	\end{tabular} \\[0.2em]
	Interviewing Android \& Flutter developers, Estimating projects, Mentoring and App development.  \\[0.4em]   
	\begin{tabular}{@{}l@{\enspace--\enspace}ll}
		\textbf{2015 Mar.} & \textbf{2017 Aug.} & Android Developer at \textbf{Corgi Inc}. \\
	\end{tabular} \\[0.2em]
	UI/UX Design, App development.   \\[0.4em]   
	\begin{tabular}{@{}l@{\enspace--\enspace}ll}
		\textbf{2012 Jan.} & Present & Self-employed \\
	\end{tabular} \\[0.2em]
	Development from scratch, some of the projects are listed on my GitHub and Play Store accounts (see links in profiles section).   

	\section*{Projects}
	Some of the projects I participated in are listed below. Most of these projects are open source.
	\subsection*{Personal projects}
	\begin{itemize}
		\item \faAndroid \enskip \href{http://artemchep.com/acdisplay/}{\textbf{AcDisplay}} --- The predecesor of Android's always on display. \\[0.2em]
		{\footnotesize \{Java, Python, IAP\} \hfill [2013--2015]} \\[0.2em]
		\textit{AcDisplay has been installed over a three million times on Google Play.} 

		\item \faAndroid \enskip \faChrome \enskip \faLinux \enskip \href{https://github.com/AChep/15puzzle}{\textbf{Game of Fifteen}} --- Best fifteen puzzle. \\[0.2em]
		{\footnotesize \{Dart, Flutter, IAP, Flutter $\Leftrightarrow$ Android, encrypt\} \hfill [2018--Present]} \\[0.2em]
		Key features: natural controls and animations, encrypted state, consistently highc complexity level of generated puzzles, dark and light theme, tablet/web/linux support. 
		
		\item \faAndroid \enskip \href{https://github.com/AChep/PocketMode}{\textbf{Pocket mode}} --- Lock your phone by hovering the proximity sensor. \\[0.2em]
		{\footnotesize \{Kotlin, Coroutines, MVVM, IAP, RenderScript\} \hfill [2018--Present]} \\[0.2em]
		Key features: modularity, customizability, dark theme support, turns screen off without requiring the pin on next unlock.   
		
		\item \faAndroid \enskip \href{https://github.com/AChep/literaryclock}{\textbf{Literary Clock}} --- Time in a form of literature quotes. \\[0.2em]
		{\footnotesize \{Kotlin, Coroutines, Firebase, MVVM, Realm\} \hfill [2018--Present]} \\[0.2em]
		Key features: syncronization with a Firebase server, local cache, widget, screen saver.  
			
		\item \faAndroid \enskip \href{https://github.com/AChep/horlogo}{\textbf{Horlogo}} --- Exemplary stylish simple watch face. \\[0.2em]
		{\footnotesize \{Kotlin, Coroutines, MVP, IAP\} \hfill [2018--Present]} \\[0.2em]
		Key features: modularity, sync between Wear OS and corresponding Android app, use of View's rending under the hood.   
		
		\item \faAndroid \enskip \href{https://play.google.com/store/apps/details?id=com.artemchep.essence}{\textbf{Essence}} --- Circular powerful watch face. \\[0.2em]
		{\footnotesize \{Kotlin, Coroutines, MVVM, RenderScript\} \hfill [2019--Present]} \\[0.2em]
		Key features: circular layout.  
		
		\item {\footnotesize \faAndroid \enskip \href{https://github.com/AChep/acpods}{\textbf{AcPods}} [2018--Present] --- Helps with intergation of Apple's AirPods. Designed and implemented the concept. Key features: modularity, hexagonal architecture, decoding of AirPods state.}   
		
		\item {\footnotesize \faAndroid \enskip \href{http://artemchep.com/horario/}{\textbf{Horario}} [2016--2018] --- Learning muliplatform platform for students that aims to improving communication between students and teachers. Designed and implemented the concept, designed the model of document-oriented database for platform, designed the logo, designed and implemented architecture to support both phones and tablets, maintainer of repository.} 
		
		\item {\footnotesize \faAndroid \enskip \href{https://forum.xda-developers.com/showthread.php?t=1188486}{\textbf{GingerDX}} [2012--2013] --- Fast, smooth and lightweight CyanogenMOD 7 based ROM with some special features. Added motion gestures to the alarm app (merged into LineageOS later on), modified system UI to be looking like Android 4.x, helped to write digitizer module to bring partial multitouch support.}
		
		\item {\footnotesize \href{https://github.com/AChep/SudokuSolver}{\textbf{SudokuSolver}} [2015--2015] --- Simple sudoku-solver that can be used to solve or help to generate sudokus.}
	\end{itemize}
	
	\subsection*{Closed Source projects}
	\begin{itemize}
		\item \faAndroid \enskip {\scshape\bfseries REDACTED} --- Multi-client app for playing videos \& games. \\[0.2em]
		{\footnotesize Android Developer, used \{Kotlin, Coroutines, Arrow, MVVM, Kodein, clean, moshi, Unity, NanoHTTPD, ExoPlayer, glide, adobe experience, IAP, datastore \& more \} \hfill [2020--Present]} \\[0.2em]
		Development from scratch. Maintained and leaded the android project. Key features: custom navigation handling, using ExoPlayer's cache to cache all content, custom views, screens \& nav graph are defined by the .json, Unity games, Web games, per-client features, exstensive use of reactive patterns and more. 
		
		\item \faAndroid \enskip \href{http://getcorgi.com/}{\textbf{Corgi for Feedly}} --- Your feed right on the lock screen. \\[0.2em]
		{\footnotesize Software Engineer \& Designer at Corgi Inc., used \{Java, Retrofit, RX, Butterknife, Glide, OkHttp, GSON\} \hfill [2015--2017]} \\[0.2em]
		Designed and developed the Android Keyguard and Notifications modules, implemented the graphical user interface, helped to design the concept of \textit{extended} lock screen, helped to maintain git-repository of the project.
		
		\item \faAndroid \enskip {\scshape\bfseries REDACTED} --- Lets you connect with people and easily exchange benefits. \\[0.2em]
		{\footnotesize Android Developer, used \{Kotlin, Coroutines, Firebase, dirty MVVM, Kodein, dirty clean, Jackson, Nitrite\} \hfill [2018--2019]} \\[0.2em]
		Maintained the project, dragastically improved the performance of the app, migrated from Kotlin 1.2 to 1.3, migrated to AndroidX, redesigned parts of the app.
		
		\item \faAndroid \enskip {\scshape\bfseries REDACTED} --- Order \& Pay for the food. \\[0.2em]
		{\footnotesize Android Developer \& Designer, used \{Java, Retrofit, OkHttp, Glide, GSON\} \hfill [2018-2018]} \\[0.2em]
		Developed a module for ordering food from the restaurants, suggested the changes in the app's design which were implemented later on. 
		
		\item \faAndroid \enskip {\scshape\bfseries REDACTED} --- Sports client management. \\[0.2em]
		{\footnotesize Android Developer, used \{Kotlin, Coroutines, Firebase, MVP, Glide\} \hfill [2018--2018]} \\[0.2em]
		Leaded the development of the app, helped to fix the existing issues with Firebase database, helped to fix the exisiting issues of the web-client written on Angular, maintaned the CI/CD server.

	\end{itemize}
	\subsection*{Profiles}
	\begin{tabular}{@{}lll}
		\faGithub & GitHub & \href{https://github.com/AChep}{\texttt{github.com/AChep}} \\
		\faLinkedin & LinkedIn & \href{https://www.linkedin.com/in/artemchep/}{\texttt{linkedin.com/in/artemchep}} \\
		\faStackOverflow & Stack Overflow & \href{https://stackoverflow.com/users/1408535/achep}{\texttt{stackoverflow.com/users/1408535/achep}} \\
		\faPlay & Play Store & \href{https://play.google.com/store/apps/dev?id=8988387544477629485}{\texttt{play.google.com/store/apps/dev?id=8988387544477629485}} \\
	\end{tabular}

	\section*{Education}
	\textbf{2014 -- 2018, 2018 -- 2020, National Technical University <<Kharkiv polytechnic institute>> Software Engineering and Management Information Technologies}. \\
	Modules included Machine learning, Fuzzy technologies, Calculus, Database modeling, Software testing, Software analysis and modeling, Requirements analysis etc.
	
	\section*{Certificates}
	\subsection*{Cambridge Certification Authority}
	\begin{itemize}
		\item \textbf{Java Level 1}, completed on 1st March 2018
		\item \textbf{Python Level 1}, completed on 1st March 2018
	\end{itemize}

	\section*{Achievements}
	Recognized Developer on \href{https://forum.xda-developers.com/member.php?u=3685328}{XDA-Developers} from 2013 to present time, Recognized Themer from 2013 to 2018.
	A prizewinner of multiple regional olympiads in Programming, Information Technology, Astronomy and Math.

\end{document}
